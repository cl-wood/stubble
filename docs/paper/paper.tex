% v2-acmsmall-sample.tex, dated March 6 2012
% This is a sample file for ACM small trim journals
%
% Compilation using 'acmsmall.cls' - version 1.3 (March 2012), Aptara Inc.
% (c) 2010 Association for Computing Machinery (ACM)
%
% Questions/Suggestions/Feedback should be addressed to => "acmtexsupport@aptaracorp.com".
% Users can also go through the FAQs available on the journal's submission webpage.
%
% Steps to compile: latex, bibtex, latex latex
%
% For tracking purposes => this is v1.3 - March 2012

\documentclass[prodmode,acmtecs]{acmsmall} % Aptara syntax

% Package to generate and customize Algorithm as per ACM style
\usepackage[ruled]{algorithm2e}
\renewcommand{\algorithmcfname}{ALGORITHM}
\SetAlFnt{\small}
\SetAlCapFnt{\small}
\SetAlCapNameFnt{\small}
\SetAlCapHSkip{0pt}
\IncMargin{-\parindent}

% Metadata Information

\def\@acmVolume{0} %the volume
\def\@acmNumber{0} %the number
\def\@acmArticle{0} %article number
\def\@articleSeq{0} %article Sequence
\def\@acmPrice{0} %article price  % Changed to 15 - June 2012 - Gerry
\def\@acmYear{2013} %the last two digits of the year, 
\def\@acmMonth{12}  %the month
\def\@journalName{COMPUTER AND NETWORK SYSTEM ADMINISTRATION} %the name of the ACM journal
\def\@journalNameShort{CNT 5605} %the acronym of the ACM journal 
\def\@journalCode{jn} %the code of the ACM journal 
\def\@permissionCodeOne{0000-0000} %the permission code of the ACM journal
\def\@doi{0000000.0000000} % These 'default' '0' values are over-ridden, during production, with 'correct' numbers entered via source .tex file - Gerry March 2012

% Document starts
\begin{document}

% Page heads
% \markboth{G. Zhou et al.}{A Multifrequency MAC Specially Designed for WSN Applications}

% Title portion
\title{The Incredible Mutating Test Files! Smart Mutation for Binary Fuzzing}
\author{Clark Wood
\affil{Florida State University}
% NOTE! Affiliations placed here should be for the institution where the
%       BULK of the research was done. If the author has gone to a new
%       institution, before publication, the (above) affiliation should NOT be changed.
%       The authors 'current' address may be given in the "Author's addresses:" block (below).
%       So for example, Mr. Abdelzaher, the bulk of the research was done at UIUC, and he is
%       currently affiliated with NASA.
}

\category{CNT 5605}{Fall 2013}{Dynamic Taint Analysis, Mutation, Fuzzing}


\acmformat{Clark Wood, Paper CNT 5605}
% At a minimum you need to supply the author names, year and a title.
% IMPORTANT:
% Full first names whenever they are known, surname last, followed by a period.
% In the case of two authors, 'and' is placed between them.
% In the case of three or more authors, the serial comma is used, that is, all author names
% except the last one but including the penultimate author's name are followed by a comma,
% and then 'and' is placed before the final author's name.
% If only first and middle initials are known, then each initial
% is followed by a period and they are separated by a space.
% The remaining information (journal title, volume, article number, date, etc.) is 'auto-generated'.

\maketitle

\section{Fuzzing Overview}

white/black/graybox, mutation/generation, Dynamic Taint Analysis, 


As programs grow larger, so do the test suites designed to exercise execution paths within the program. White box approaches have the advantage of source code, allowing developers to write test cases which exercise paths in a higher level language. With black box approaches, however, test cases would typically target assembly block-level code paths [A,...NEEDED].

\subsection{Dynamic Taint Analysis}

Dynamic taint analysis (DTA) is the process of locating sources of taint and following their propagation through a binary to data sinks. Since, for exploitation, bugs must be triggered by user input, DTA usually identifies all user input (files, command line arguments, UI interactions, ...) as tainted, and marks it accordingly.

The granularity of taint marking is variable. Assigning everything from a generic binary value of tainted/not tainted to bit-level marking of which part of user input affected which variable is conceivable, although finer-grained taint analysis is more computationally expensive and difficult to determine [CITE]. Consider the following pseudocode:
\begin{verbatim} 
x = userInput & 1
if x % 2 == 0:
	vuln()				
\end{verbatim}

TODO
Taint can propagate via explicit or implicit flows (Figure 1). Explicit flows involve a tainted variable, x, which is used in an assignment expression to compute a new variable, y. In this situation, x taints y, and if y is involved in any further assignment to a variable z, then x taints z by transitivity. In contrast, implicit data flows ... [D]. 

Because of subtle implicit flow cases referenced by Clause, J. et all [D], which can be difficult to spot (Page 2, Figure 2b). Implicit data flows are not always considered by dynamic tainting techniques [E].

\begin{verbatim}
# Explicit taint

# Implicit taint
\end{verbatim}
  


Propagation

\section{Machine Learning}
While research has applied machine learning techniques to successfully extrapolate vulnerability patterns, previous efforts have required source code and training examples containing known vulnerabilities to drive learning [F]. In contrast, we implement ...

\section{Dynamic Binary Instrumentation with Intel PIN}

Pintools

\section{Ways to mutate test cases}

Machine learning in lieu of symbolic execution?

\section{Implementation}

	\subsection{Pintools for Finding Vulnerable Conditions}

	\subsection{Mutation Algorithm}


Fuzzing optimization techniques are evolving to deal with the path explosion problem [CITES] in various ways, focusing on ...

Researchers at VUPEN Security mention the value of discovering test case sets of equivalent coverage, but smaller size, via test suite reduction algorithms [A].

\section{Future Work}


\section{References}

[A] Bekrar, Sofia, et al. "A taint based approach for smart fuzzing." Software Testing, Verification and Validation (ICST), 2012 IEEE Fifth International Conference on. IEEE, 2012.
[B] DeMott, J., Enbody, R., and Punch, W. "Revolutionizing the Field of Grey-box Attack Surface Testing with
Evolutionary Fuzzing", BlackHat and Defcon 2007.
[C] Schwartz, E.J., Avgerinos, T., Brumley, D. "All You Ever Wanted to Know About Dynamic Taint Analysis and Forward Symbolic Execution (but might have been afraid to ask)." 2010 IEEE Symposium on Security and Privacy.
[D] Clause, J. Li, W., Orso, A. "Dytan: a generic dynamic taint analysis framework". 2007 Int'l symposium on Software testing and analysis. ACM, 2007.
[E] ... "Beyond Instruction Level Taint Propagation".
[F] Yamaguchi, Fabian, Felix Lindner, and Konrad Rieck. "Vulnerability extrapolation: assisted discovery of vulnerabilities using machine learning." Proceedings of the 5th USENIX conference on Offensive technologies. USENIX Association, 2011.

\end{document}

% End of v2-acmsmall-sample.tex (March 2012) - Gerry Murray, ACM


